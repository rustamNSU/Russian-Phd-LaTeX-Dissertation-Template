\documentclass[a4paper,12pt]{article}

\usepackage[T2A]{fontenc}
\usepackage[utf8]{inputenc}
\usepackage[english,russian]{babel}
\usepackage{amssymb,amsfonts,amsmath,mathtext}
\usepackage{cite,enumerate,float,indentfirst}
\usepackage{geometry}
\geometry{left=3cm}
\geometry{right=1cm}
\geometry{top=2cm}
\geometry{bottom=2cm}

\renewcommand{\baselinestretch}{1.4}   % Полуторный межстрочный интервал
\parindent 1.27cm   % Абзацный отступ

\begin{document}

\begin{center}
    \textbf{
        \hfill \break
        Реферат\\
        \hfill \break
        Построение матрицы жесткости для слоистой среды в методе разрывных смещений \\
    }
\end{center}

Выпускная квалификационная работа магистра содержит 26 страниц формата A4, включающие 3 приложения, 12 иллюстраций и таблиц, 10 использованных источников.

Ключевые слова: Матрица упругости, матрица жесткости, Planar3d ILSA, метод разрывных смещений, неоднородность модулей упругости, слоистая среда, гидроразрыв пласта.

Дипломная работа посвящена построению матрицы жесткости для слоистой среды в методе разрывных смещений. В рамках дипломной работы был реализован метод численного построения матрицы жесткости для слоистой среды в методе разрывных смещений, выполнено встраивание данного алгоритма в модель гидроразрыва пласта (ГРП) Planar3d ILSA, реализованной ранее в лаборатории цифровых и интеллектуальных систем добычи углеводородов в Институте гидродинамики им. М. А. Лаврентьева СО РАН. В работе проведен параметрический анализ задачи и показано существенное влияние неоднородности модулей упругости на геометрию трещины ГРП.

В первом разделе приводится математическая формулировка задачи моделирования плоской трещины ГРП в слоистой среде и ее дискретизация методом разрывных смещений.

Во втором разделе описывается алгоритм численного построения матрицы жесткости для слоистой среды с неоднородностью по модулям упругости.

В третьем разделе проведена верификация метода путем сравнения с известными литературными данными и точными решениями. Проведен анализ влияния учета слоистой структуры пласта на конечную геометрию трещины.
Изучено влияние тонких жестких пропластков на раскрытие трещины, что важно при расчете переноса расклинивающего агента (проппанта) и других компонент жидкости по трещине ГРП.


\thispagestyle{empty} % выключаем отображение номера для этой страницы
\end{document}
