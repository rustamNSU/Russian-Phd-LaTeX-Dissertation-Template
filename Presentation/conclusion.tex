\section*{Заключение}
\begin{frame}
    \frametitle{Заключение}
    \textbf{Основные результаты работы} заключаются в следующем:
    \begin{enumerate}
        \item Реализован алгоритм построения численной матрицы упругости для МРС на языке~C++.
        \item Выполнено сопряжение данного алгоритма с моделью  Planar3d ILSA.
        \item Проведена верификация метода, показано влияние учета слоистой структуры пласта на конечную геометрию трещины.
        \item Показано, что тонкие жесткие пропластки не ограничивают рост трещины, однако влияют на раскрытие трещины в них.
    \end{enumerate}

    \textbf{Практическая значимость работы} заключается в возможности применения полученных результатов для
    \begin{itemize}
        \item моделирования процесса ГРП в слоистых средах,
        \item изучения свойств слоистых материалов на основе приведенного метода.
    \end{itemize}
\end{frame}

% \begin{frame} % публикации на одной странице
% % \begin{frame}[t,allowframebreaks] % публикации на нескольких страницах
%     \frametitle{Основные публикации}
%     \nocite{vakbib1}%
%     \nocite{vakbib2}%
%     %
%     %% authorwos
%     \nocite{wosbib1}%
%     %
%     %% authorscopus
%     \nocite{scbib1}%
%     %
%     %% authorconf
%     \nocite{confbib1}%
%     \nocite{confbib2}%
%     %
%     %% authorother
%     \nocite{bib1}%
%     \nocite{bib2}%
%     \ifnumequal{\value{bibliosel}}{0}{
%         \insertbiblioauthor
%     }{
%         \printbibliography%
%     }
% \end{frame}
