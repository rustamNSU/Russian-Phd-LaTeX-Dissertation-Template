\section{Заключение}
\begin{frame}
    \frametitle{Заключение}
    \textbf{Основные результаты работы} заключаются в следующем:
    \begin{itemize}
        \item Реализован алгоритм построения численной матрицы жесткости для метода разрывных смещений в случае неоднородных по модулю упругости слоев на языке программирования~C++.
        \item Выполнено встраивание данного алгоритма в модель раскрытия трещины гидроразрыва пласта Planar3D ILSA.
        \item Проведена верификация метода путем сравнения с известными литературными данными и точными решениями.
        \item Проведен анализ влияния учета слоистой структуры пласта на конечную геометрию трещины.
        \item Показано, что тонкие жесткие пропластки не существенно ограничивают рост трещины в вертикальном направлении, однако влияют на раскрытие трещины в них, что может сильно сказаться на распространение проппанта и конечной геометрии трещины ГРП. 
    \end{itemize}
\end{frame}

% \begin{frame} % публикации на одной странице
% % \begin{frame}[t,allowframebreaks] % публикации на нескольких страницах
%     \frametitle{Основные публикации}
%     \nocite{vakbib1}%
%     \nocite{vakbib2}%
%     %
%     %% authorwos
%     \nocite{wosbib1}%
%     %
%     %% authorscopus
%     \nocite{scbib1}%
%     %
%     %% authorconf
%     \nocite{confbib1}%
%     \nocite{confbib2}%
%     %
%     %% authorother
%     \nocite{bib1}%
%     \nocite{bib2}%
%     \ifnumequal{\value{bibliosel}}{0}{
%         \insertbiblioauthor
%     }{
%         \printbibliography%
%     }
% \end{frame}
