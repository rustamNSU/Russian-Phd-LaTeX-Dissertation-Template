\chapter*{Заключение}                       % Заголовок
\addcontentsline{toc}{chapter}{Заключение}
\label{sec:conclusion} 

\textbf{Основные результаты работы} заключаются в следующем:
\begin{enumerate}
    \item Реализован алгоритм построения численной матрицы упругости для метода разрывных смещений в случае неоднородных по модулю упругости слоев на языке программирования~C++.
    \item Выполнено встраивание данного алгоритма в модель раскрытия трещины гидроразрыва пласта Planar3d ILSA.
    \item Проведена верификация метода путем сравнения с известными литературными данными и точными решениями.
    \item Проведен анализ влияния учета слоистой структуры пласта на конечную геометрию трещины.
    \item Показано, что тонкие жесткие пропластки не существенно ограничивают рост трещины в вертикальном направлении, однако влияют на раскрытие трещины в них, что может сильно сказаться на распространение проппанта и конечной геометрии трещины ГРП. 
\end{enumerate}