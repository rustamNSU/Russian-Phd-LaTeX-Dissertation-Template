\chapter*{Заключение}                       % Заголовок
\addcontentsline{toc}{chapter}{Заключение}
\label{sec:conclusion} 

\textbf{Основные результаты работы} заключаются в следующем:
\begin{enumerate}
    \item Реализован алгоритм построения численной матрицы упругости для МРС в случае неоднородных по модулю упругости слоев на языке~C++.
    \item Выполнено сопряжение данного алгоритма с моделью раскрытия трещины гидроразрыва Planar3d ILSA.
    \item Проведена верификация метода путем сравнения с известными литературными данными и точными решениями.
    \item Проведен анализ влияния учета слоистой структуры пласта на конечную геометрию трещины.
    \item Показано, что тонкие жесткие пропластки не существенно ограничивают рост трещины, однако влияют на раскрытие трещины в них.
\end{enumerate}

\textbf{Практическая значимость работы} заключается в возможности применения полученных результатов для
\begin{itemize}
    \item моделирования процесса ГРП в слоистых средах,
    \item изучения свойств слоистых материалов на основе приведенного метода.
\end{itemize}