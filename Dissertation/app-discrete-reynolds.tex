\chapter{Дискретизация уравнения Рейнольдса}
\label{app:discrete-reynolds}

Интегрируя уравнение Рейнольдса~\eqref{eq:reynolds_equation} по времени от $t-\Delta t$ до $t$, где $\Delta t$ --- шаг по времени, и элементу $\mathcal{A}_{i,j}$, получим дискретизацию по методу конечных объемов
\begin{equation}
    \label{eq:discrete-reynolds}
    w_{i, j}(t) - w_{i, j}(t-\Delta t) = \Delta t[Ap]_{i, j} + \Delta t Q_{i, j} - \Delta \mathcal{L}_{i, j},
\end{equation}
где $\Delta t Q_{i, j}$ отвечает за закачку, $\Delta \mathcal{L}_{i, j}$ --- утечка жидкости в пласт с момента времени $t-\Delta t$ до $t$. $[Ap]_{i, j}$ --- это потоки жидкости через границу элемента $\mathcal{A}_{i,j}$, которые вычисляются как

\begin{equation}
    \begin{split}
        [Ap]_{i, j} &= \frac{1}{\Delta x} \left[M_{i+\frac{1}{2},j} \frac{p_{i+1,j}-p_{i,j}}{\Delta x} + M_{i-\frac{1}{2},j} \frac{p_{i-1,j}-p_{i,j}}{\Delta x} \right] + \\
        &= \frac{1}{\Delta y} \left[M_{i,j+\frac{1}{2}} \frac{p_{i,j+1}-p_{i,j}}{\Delta y} + M_{i,j+\frac{1}{2}} \frac{p_{i-1,j}-p_{i,j}}{\Delta y} \right],
    \end{split}
\end{equation}
где соответствующие мобильности жидкости определяются как
\begin{equation}
    M_{i \pm \frac{1}{2},j} = \frac{w^3_{i \pm 1,j} + w^3_{i,j}}{2\mu'}.
\end{equation}

Задача численного моделирования раскрытия трещины ГРП состоит в решении системы уравнений \eqref{eq:discrete-reynolds} и \eqref{eq:discrete_elasticity}.
